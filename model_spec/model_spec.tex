\RequirePackage[l2tabu,orthodox]{nag}
\documentclass[11pt,letterpaper]{article}
\usepackage[T1]{fontenc}
\usepackage[utf8]{inputenc}
\usepackage{crimson}
\usepackage{helvet}
\usepackage[strict,autostyle]{csquotes}
\usepackage[USenglish]{babel}
\usepackage{microtype}
\usepackage{authblk}
\usepackage{booktabs}
\usepackage{caption}
\usepackage{endnotes}
\usepackage{geometry}
\usepackage{graphicx}
\usepackage{hyperref}
\usepackage{natbib}
\usepackage{rotating}
\usepackage{setspace}
\usepackage{titlesec}
\usepackage{url}
\usepackage{amssymb,amsmath}

% location of figure files, via graphicx package
\graphicspath{{./figures/}}

% configure the page layout, via geometry package
\geometry{
	paper=letterpaper,
	top=4cm,
	bottom=4cm,
	left=4cm,
	right=4cm}
\setstretch{1.02}
\clubpenalty=10000
\widowpenalty=10000

% set section/subsection headings as the sans serif font
\titleformat{\section}{\normalfont\sffamily\large\bfseries}{\thesection.}{0.3em}{}
\titleformat{\subsection}{\normalfont\sffamily\small\bfseries}{\thesubsection.}{0.3em}{}

% make figure/table captions sans-serif small font
\captionsetup{font={footnotesize,sf},labelfont=bf,labelsep=period}

% configure pdf metadata and link handling
\hypersetup{
	pdfauthor={Francisco Rowe},
	pdftitle={xxx},
	pdfsubject={xxx},
	pdfkeywords={xxx},
	pdffitwindow=true,
	breaklinks=true,
	colorlinks=false,
	pdfborder={0 0 0}}

\title{ \footnote{\textbf{Citation}: Rowe, F. 2021. xxx.}}

\author[1]{Francisco Rowe \thanks{\textit{Corresponding author}: F.Rowe-Gonzalez@liverpool.ac.uk}}
\author[2]{xxx}

\affil[1]{Geographic Data Science Lab, Department of Geography and Planning, University of Liverpool, Liverpool, United Kingdom}
\affil[2]{xxx}

\date{}

\begin{document}

\maketitle


\newpage

\section*{Model Specification}

This note seeks to document the model specification we would ideally like to estimate and its key components and reference terms. 

\begin{equation}\label{m1}
  \begin{split}
&p_{ijt} = \alpha_{ij} + \gamma_{t} + FB_{ijt} + \varepsilon_{ijt} \\
&\alpha_{ij} = \alpha_{i} + \alpha_{j} + \beta_{1}d_{ij} + \beta_{2}c_{ij} + u_{ij} \\
&\alpha_{i} = \delta_{i} + \delta_{i1}pop_{i}  + F_{i} + u_{i} \\
&\alpha_{j} = \delta_{j} + \delta_{j1}pop_{j}  + F_{j} + u_{j} \\
&\gamma_{t} = \delta_{t} + \delta_{t1}day +  \delta_{t2}ti + \delta_{t3}w  + u_{t} \\
&FB_{ijt} = fb_{0} +  fb_{ijt}z_{ijt} + fb_{ijt}au_{ijt} + u_{ijt} \\
  \end{split}
\end{equation}

$p_{ijt}$: proportion of people moving from tile $i$ to tile $j$ at time $t$;\\
$\alpha_{ij}$: component relating to tiles of origin $i$ and destionation $j$;\\
$\gamma_{t}$: component relating to temporal variation;\\
$FB_{ijt}$: component relating to variation in Facebook data quality;\\
$\alpha_{i}$: factors relating to tiles of origin $i$;\\
$\alpha_{j}$: factors relating to tiles of origin $j$;\\
$\beta_{1}$: coefficient capturing the deterring effect of distance $d_{ij}$;\\
$\beta_{2}$: coefficient capturing the variation in mobility intensity across tiles in different functional areas $c_{ij}$;\\
$\delta_{i}$: average level of mobility related to tiles of origin $i$;\\
$\delta_{i1}$: coefficient capturing the relationship between mobility intensity and population at origin tile $pop_{i}$;\\
$F_{i}$: coefficient capturing the relationship between mobility intensity and additional attributes at origin $i$ eg. unemployment, housing affordability, etc.\\ 
$\delta_{j}$: average level of mobility related to tiles of destination $j$;\\
$\delta_{j1}$: coefficient capturing the relationship between mobility intensity and population at destination tile $pop_{j}$;\\
$F_{j}$: coefficient capturing the relationship between mobility intensity and additional attributes at destination $j$ eg. unemployment, housing affordability, etc.\\ 
$\delta_{t}$: average level of mobility related to time;\\
$\delta_{t1}$: coefficient capturing the daily $day$ variability in mobility;\\
$\delta_{t2}$: coefficient capturing the variability in mobility according to the 8 hour time interval $ti$ in which the Facebook data are organised; that is, 0am, 8am and 4pm;\\
$\delta_{t3}w$: coefficient capturing the variability in mobility during week days and weekends $w$;\\
$\varepsilon$ and $u$ correspond to the error terms;

\section*{Estimation}

The plan is to estimate Equation~\ref{m1} in a fractional regression framework \citep[see][]{papke1996econometric}. 
In this framework errors are assumed to follow a logistic distribution and quasi maximum likelihood procedure is used to estimate the set of regression coefficients.
The variable identifying pair combinations of functional areas for tiles of origin and destination (i.e. $c_{ij}$) can enter the model as a dummy variable or as a random effect in a multilevel framework.
Time can also be modelled in different ways (i.e. including autoregresive terms of $p_{ijt}$, autoregresive terms of errors and splines) and there may be different components which may need to be taken into consideration (i.e. seasonability and autocorrelation)

\newpage

% print the bibliography
\setlength{\bibsep}{0.00cm plus 0.05cm} % no space between items
\bibliographystyle{apalike}
\bibliography{itinerant_model}

\end{document}
